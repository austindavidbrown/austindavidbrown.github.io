\documentclass[12pt, reqno]{amsart}
\usepackage[utf8]{inputenc}
\usepackage{amsmath, amsfonts, amssymb, amsthm, setspace, hyperref}
\numberwithin{equation}{section}
\parindent 0mm
\onehalfspacing
\addtolength{\textwidth}{2 truecm}
\setlength{\hoffset}{-1 truecm}

\newcommand{\e}{\epsilon}
\newcommand{\R}{\mathbb{R}}
\newcommand{\N}{\mathbb{N}}
\newcommand{\Z}{\mathbb{Z}}
\newcommand{\w}{\omega}
\let\oldemptyset\emptyset
\let\emptyset\varnothing

\newcommand{\intersect}{\bigcap}
\newcommand{\union}{\bigcup}
\newcommand{\disjointunion}{\coprod} 
\newcommand{\powerset}{\mathcal{P}}
\newcommand{\dist}{\sim}
\renewcommand{\P}{P}
\newcommand{\E}{\text{E}}
\newcommand{\Var}{\text{Var}}
\newcommand{\given}{\mid}

\begin{document}

\title{Casella and Berger Notes and Exercises -  Chapter 2}
\author{Austin David Brown}
\maketitle

\section*{Chapter 2 Exercises}

Using this problem set:
http://www.stat.ufl.edu/~jhobert/sta6326

Chapter 2: 1, 2, 3, 4, 9, 11, 12, 13, 14, 16, 17

Chapter 2: 22, 23, 25, 26, 27, 30, 31, 33, 36, 38

{\bf (2.1)\rule{\textwidth}{.5pt}}

Apply change of variables for each.

Check that the given PDF's are PDF's and then the result follows.

{\bf (2.2)\rule{\textwidth}{.5pt}}

Apply change of variables again. Straight forward.

{\bf (2.3)\rule{\textwidth}{.5pt}}

This one tripped me up for some reason.
Let $B$ be a measurable set in the image of $Y$.
Since $X$ is discrete $B = \{b\}$.
Then $P(Y \in B) \iff P(Y = b) \iff P(X = b / (1 - b)) = f_X(b/(1 - b))$

It is much straight forward to just change the variables this way in my opinion instead of using distribution functions or something.

{\bf (2.4)\rule{\textwidth}{.5pt}}

This is clear by the definition and additivity of the integral.

{\bf (2.9)\rule{\textwidth}{.5pt}}

The distribution function.

{\bf (2.11)\rule{\textwidth}{.5pt}}

TODO COME BACK TO THIS

{\bf (2.12)\rule{\textwidth}{.5pt}}

{\bf (2.14)\rule{\textwidth}{.5pt}}

Very nice application of Fubini's Theorem.

{\bf (2.16)\rule{\textwidth}{.5pt}}

TODO

{\bf (2.17)\rule{\textwidth}{.5pt}}

Straight forward.

{\bf (2.22)\rule{\textwidth}{.5pt}}

REDO Tricky. Better to transform into the Gamma function.

{\bf (2.23)\rule{\textwidth}{.5pt}}

Kind of a change of variables in integration problem.

{\bf (2.25)\rule{\textwidth}{.5pt}}

For some reason I got messed up on this change of variables.

{\bf (2.26)\rule{\textwidth}{.5pt}}

REDO first parts are tricky.

{\bf (2.27)\rule{\textwidth}{.5pt}}

TODO

{\bf (3.30, 3.31, 3.33, 3.36, 3.38) \rule{\textwidth}{.5pt}}

These are all pretty straight forward power series transformation stuff.

\section*{Notes}

8 / 20
\rule{\textwidth}{.5pt}

8 / 21
\rule{\textwidth}{.5pt}

Changing the variables in measures should really go like this:

For any measurable set $B$,
$P(Y \in B) \iff \cdots$
And use change of variables.
Example:
$P(-t \le Y \le t)$ for continous
or $P(X = t)$ for discrete since these are the measurable sets.

USE THIS FORMULA WHEN CHANGING VARIABLES
$P(X \in B) \iff P(b_1 \le X \le b_2)$ or $P(X \in B) \iff P(X = t)$


8 / 22
\rule{\textwidth}{.5pt}

I really forgot how to use by parts.
It really follows from FTC2 (See Munkres).


\end{document}
