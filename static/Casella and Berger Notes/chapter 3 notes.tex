\documentclass[12pt, reqno]{amsart}
\usepackage[utf8]{inputenc}
\usepackage{amsmath, amsfonts, amssymb, amsthm, setspace, hyperref}
\numberwithin{equation}{section}
\parindent 0mm
\onehalfspacing
\addtolength{\textwidth}{2 truecm}
\setlength{\hoffset}{-1 truecm}

\newcommand{\e}{\epsilon}
\newcommand{\R}{\mathbb{R}}
\newcommand{\N}{\mathbb{N}}
\newcommand{\Z}{\mathbb{Z}}
\newcommand{\w}{\omega}
\let\oldemptyset\emptyset
\let\emptyset\varnothing

\newcommand{\intersect}{\bigcap}
\newcommand{\union}{\bigcup}
\newcommand{\disjointunion}{\coprod} 
\newcommand{\powerset}{\mathcal{P}}
\newcommand{\dist}{\sim}
\renewcommand{\P}{P}
\newcommand{\E}{\text{E}}
\newcommand{\Var}{\text{Var}}
\newcommand{\given}{\mid}

\begin{document}

\title{Casella and Berger Notes and Exercises -  Chapter 3}
\author{Austin David Brown}
\maketitle

\section*{Chapter 3 Exercises}

Using this problem set:
http://www.stat.ufl.edu/~jhobert/sta6326

Chapter 3: 2, 4, 7, 10, 13, 14, 19, 20, 23, 24, 28, 37, 38, 42, 45, 46

{\bf (3.2)\rule{\textwidth}{.5pt}}

REDO tricky.

{\bf (3.4)\rule{\textwidth}{.5pt}}

(a) is geometric

(b) is hypergeometric?

{\bf (3.7)\rule{\textwidth}{.5pt}}

 REDO messed this one up bad

{\bf (3.10)\rule{\textwidth}{.5pt}}

TODO

{\bf (3.14)\rule{\textwidth}{.5pt}}

Skipping

{\bf (3.19)\rule{\textwidth}{.5pt}}

{\bf (3.38)\rule{\textwidth}{.5pt}}

TODO

{\bf (3.42)\rule{\textwidth}{.5pt}}

(a) Pretty straight forward with set theory.

(b) I think this should say image space or just whatever space the scaling parameter is in.

{\bf (3.45)\rule{\textwidth}{.5pt}}

Basically repeat the Chebyschev inequality proof.

{\bf (3.46)\rule{\textwidth}{.5pt}}

Straight forward computation.

\section*{Notes}

8 / 22
\rule{\textwidth}{.5pt}

Left off on page 151

8 / 24
\rule{\textwidth}{.5pt}

\end{document}
