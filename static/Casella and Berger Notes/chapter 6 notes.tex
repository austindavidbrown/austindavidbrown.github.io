\documentclass[12pt, reqno]{amsart}
\usepackage[utf8]{inputenc}
\usepackage{amsmath, amsfonts, amssymb, amsthm, setspace, hyperref}
\numberwithin{equation}{section}
\parindent 0mm
\onehalfspacing
\addtolength{\textwidth}{2 truecm}
\setlength{\hoffset}{-1 truecm}

\newcommand{\e}{\epsilon}
\newcommand{\R}{\mathbb{R}}
\newcommand{\N}{\mathbb{N}}
\newcommand{\Z}{\mathbb{Z}}
\newcommand{\w}{\omega}
\let\oldemptyset\emptyset
\let\emptyset\varnothing
\let\phi\varphi

\newcommand{\intersect}{\bigcap}
\newcommand{\union}{\bigcup}
\newcommand{\disjointunion}{\coprod} 
\newcommand{\powerset}{\mathcal{P}}
\newcommand{\dist}{\sim}
\renewcommand{\P}{P}
\newcommand{\E}{\text{E}}
\newcommand{\Var}{\text{Var}}
\newcommand{\given}{\mid}
\newcommand{\mean}{\bar}

\begin{document}

\title{Casella and Berger Notes and Problems -  Chapter 6}
\author{Austin David Brown}
\maketitle

Chapter 6: 1, 2, 3, 5, 7, 10, 13, 14, 17, 19, 21, 22, 24, 30

\section{Problems}

(6.1)

Use Factorization Theorem.

(6.2)
TODO

(6.3)
TODO

(6.5)
TODO

(6.7)
TODO

(6.10)
TODO

(6.13)
TODO

(6.14)
TODO

(6.17)
TODO

(6.19)

Some algebra shows g is not constant in respect to $p$.

(6.21)

(a) Look at the map $F(v) = (p_{X(-1)} v_1, p_{X(0)} v_2, p_{X(1)} v_3)$ on the space $g(-1) \times g(0) \times g(1)$. The map is not injective, and hence $g$ is not $0$.

(b) Look at the map $F(v) = (p_{X(0)} v_2, p_{X(1)} v_3)$ on the space $g(0) \times g(1)$. The map is injective, and hence $g$ is $0$.

(c) Yes, by algebra. The exponential family in part (a) is not full rank in the sense of exponential families is where the problem lies.

(6.24)
I believe this failed for $\lambda = 0.$

(6.30)
TODO

\section{Notes}

8 / 8
\rule{\textwidth}{.5pt}

8 / 9
\rule{\textwidth}{.5pt}

8 / 10
\rule{\textwidth}{.5pt}

The definiitons:
Exponential Family
Sufficient statistic
Minimal sufficient statistic
complete statistic
Factoring Theorem
Basu Theorem
Complete Statistic Thm for Exponential Families
Ancillary Statistic
MLE statistic

Examples of these:

Really the normal distribution is Gamma with change of variables. So it is obvious that things will have a normal distribution becuase gamma is a generalization of a factorial.

Common order statistics:
max, min, range, median

8 / 11
\rule{\textwidth}{.5pt}

Exponential familes

The normal and these are based off of the gamma distribution.

The normal distribution is really just the gamma function.

All of these exponential familes originate with the gamma function which originates with factorial which represents permutations.

Def. Complete Statistic
X is complete if the kernel of the linear map E is 0 from the space of $A \circ imX \to \R^n$

This is really a better definition.

I think Lp spaces are going to lead to some more insight into what is going on with some of this stuff.

\begin{thebibliography}{1}

\bibitem{durett}
Casella and Berger. \textit{Statistical Inference}.
\end{thebibliography}

\end{document}
