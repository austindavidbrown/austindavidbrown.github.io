\documentclass[12pt, reqno]{amsart}
\usepackage[utf8]{inputenc}
\usepackage{amsmath, amsfonts, amssymb, amsthm, setspace, hyperref}
\numberwithin{equation}{section}
\parindent 0mm
\onehalfspacing
\addtolength{\textwidth}{2 truecm}
\setlength{\hoffset}{-1 truecm}

\newcommand{\e}{\epsilon}
\newcommand{\R}{\mathbb{R}}
\newcommand{\N}{\mathbb{N}}
\newcommand{\Z}{\mathbb{Z}}
\newcommand{\w}{\omega}
\let\oldemptyset\emptyset
\let\emptyset\varnothing

\newcommand{\intersect}{\bigcap}
\newcommand{\union}{\bigcup}
\newcommand{\disjointunion}{\coprod} 
\newcommand{\powerset}{\mathcal{P}}
\newcommand{\dist}{\sim}
\renewcommand{\P}{P}
\newcommand{\E}{\text{E}}
\newcommand{\Var}{\text{Var}}
\newcommand{\given}{\mid}

\begin{document}

\title{Casella and Berger Notes and Exercises -  Chapter 1}
\author{Austin David Brown}
\maketitle

\section*{Chapter 1 Exercises}

Using this problem set:
http://www.stat.ufl.edu/~jhobert/sta6326

Chapter 1: 1, 2, 4, 13, 14, 16, 18, 21, 22, 23, 24, 27, 33

Chapter 1: 35, 38, 39, 44, 46, 47, 49, 50, 51, 52, 54

{\bf (1.1), (1.2), (1.4), (1.13), (1.14) \rule{\textwidth}{.5pt}}

These are straight forward.

{\bf (1.16)\rule{\textwidth}{.5pt}}

These follow from multiplication rule.

{\bf (1.18)\rule{\textwidth}{.5pt}}

I got
\[
\frac{n \binom{n - 1}{n - 2} }{ \binom{2n - 1}{n} },
\]
which I guess is incorrect.

{\bf (1.21)\rule{\textwidth}{.5pt}}

TODO tough one.

{\bf (1.22)\rule{\textwidth}{.5pt}}

(a)

Let $S$ $\equiv$ 180 days chosen from 366.
Then $|S| = \binom{366}{180}$.
Also, $180/12 = 15$.
Let $E$ $\equiv$ 180 days distributed amongst the 12 months equally.
We use Vandermonde's identity applied to each month as a group.
By Vandermonde's identity $12$ times with $366 = 31 + \cdots 31$ and $180 = k_1 + \cdots k_{12}$, we have
\[
\binom{366}{180}
= \sum_{k_1 = 0}^{31} \cdots \sum_{k_{12} = 0}^{31} \binom{31}{k_1} \cdots \binom{31}{k_{12}}.
\]
Hence,
\[
P(E) = \frac{\binom{31}{15} \cdots \binom{31}{k_{15}}}{\binom{366}{180}}.
\]

(b)
TODO not sure how to count this one.

{\bf (1.23)\rule{\textwidth}{.5pt}}

I think this works.

Let $S \equiv$ $2n$ flips from the $2$ people.
Then $|S| = 2^n 2^n$.
Let $E \equiv$ both people flip the same number of heads.
There are 2n total heads and we want to choose all n-subsets of them.
So, $\binom{n + n}{n}$ is the count of n-subsets of heads from a total of 2n heads.
Hence,
\[
P(E) = \frac{\binom{n + n}{n}}{2^n 2^n}.
\]


{\bf (1.24)\rule{\textwidth}{.5pt}}

This is a geometric distribution problem.

Need some more practice with this distribution.

{\bf (1.27)\rule{\textwidth}{.5pt}}

TODO Skipping these until I study more combinatorics.

{\bf (1.33)\rule{\textwidth}{.5pt}}

Partition the space by Males and Females and then use the definition of conditional probability.

\begin{align*}
P(\text{Male} \given \text{Colorblind})
&= \frac{P( \{ \text{Male} \} \intersect \{ \text{Colorblind}) \} }{ P( \{ \text{Colorblind}) \} } \\
&= \frac{P( \{ \text{Male} \} \intersect \{ \text{Colorblind}) \} }{ P( \{ \text{Male} \} \intersect \{ \text{Colorblind}) \} + P( \{ \text{Female} \} \intersect \{ \text{Colorblind}) \}}.
\end{align*}

{\bf (1.35) \rule{\textwidth}{.5pt}}

This follows straight from the definition of a measure.
Let $P : \Omega \to [0, 1]$ be a probability measure.
Let $B \in \Omega$ such that $B > 0.$
Let $(E_k)_{k \in \N}$ be disjoint sets in $\Omega.$
Then
$P(\emptyset \given B) = \emptyset$.
Since $P$ is a measure, we have
\[
P(\union_{k \in \N} E_k \given B)
= \frac{P(\union_{k \in \N} E_k \intersect B)}{P(B)}
= \sum_{k \in \N}  \frac{P(E_k \intersect B)}{P(B)}
=\sum_{k \in \N}  P(E_k \given B).
\]
Therefore, it is a measure.

{\bf (1.38) \rule{\textwidth}{.5pt}}

All of these follow straight from the definitions.

{\bf (1.39) \rule{\textwidth}{.5pt}}

Follows straight from the definitions

{\bf (1.44) \rule{\textwidth}{.5pt}}

Let $X$ be an RV that measures \# of correct answers. Then $X$ is $bin$, and
\[
P(10 \le X)
= P(10 \le X \le 20)
= \sum_{k = 10}^{20} \binom{20}{k} (1/4)^{k} (3/4)^{20 - k}
\]

{\bf (1.46) \rule{\textwidth}{.5pt}}

TODO tough one.

{\bf (1.47) \rule{\textwidth}{.5pt}}

These are all straight forward unless you think it said pdf like I did.

{\bf (1.49) \rule{\textwidth}{.5pt}}

Follows directly from definitions using compliments.

Let $t \in \R.$
If $F_X(t) \le F_Y(t)$,
\[
P(Y \in \R) - F_Y(t) \le P(X \in \R) - F_Y(t)
\iff P(Y > t) \le P(X > t).
\]
The second part follows from the second part of the definition.

{\bf (1.50) \rule{\textwidth}{.5pt}}

Using discrete calculus,
\[
\sum_{k = 1}^n t^{k - 1}
= \sum_{1 \le k \le n + 1} \Delta_k \frac{1}{t - 1}  t^{k - 1}
= \frac{t^n - 1}{t - 1}.
\]

Discrete solutions are so neat.

{\bf (1.51) \rule{\textwidth}{.5pt}}

Since we are choosing without replacement from 2 groups, this is a hypergeometric problem.

By Vandermode's identity,
\[
\binom{30} {5}
= \sum_{k = 1}^5 \binom{5}{k} \binom{25}{5 - k}
\]
iff
\[
1
= \binom{30} {5}^{-1} \sum_{k = 1}^5 \binom{5}{k} \binom{25}{5 - k}
\]
yields the pmf and cdf.

{\bf (1.52) \rule{\textwidth}{.5pt}}

Follows directly from definitions.

{\bf (1.54) \rule{\textwidth}{.5pt}}

Both of these are easy.

(a) Use Riemann-Lebesgue for the endpoints, and FTC.

(b) Use additivity of the integral.













\section*{Notes}

7 / 30
\rule{\textwidth}{.5pt}

8 / 1
\rule{\textwidth}{.5pt}

On section 1.4 currently

discrete distributions:
characteristic functions
geometric
binomial

8 / 2
\rule{\textwidth}{.5pt}


8 / 3
\rule{\textwidth}{.5pt}

8 / 4
\rule{\textwidth}{.5pt}

Difference operators properties are awesome!

\url{en.wikipedia.org/wiki/Finite_difference#Rules_for_calculus_of_finite_difference_operators}

In fact, this is much more enlightening than just using the geometric series.

Finite difference calculus

Hypergeometric and Binomial distributions are closely related.


8 / 19
\rule{\textwidth}{.5pt}

Ok I read chapter 1 completely now.

Need to do the problem set and move on to chapter 2

8 / 20
\rule{\textwidth}{.5pt}

Problem set

Memorize and understand the main Discrete Distributions:
------------
Discrete Uniform
Bournoulli, Binomial, Poisson, Multinomial
Hypergeometric, Multivariate Hypergeometric distribution
Geometric, Negative Binomial

These are really fundamental::
-----------
Multinomial and Multivariate Hypergeometric.
Multinomial - replacement
Multivariate Hypergeometric - no replacement

See Wikipedia for Hypergeometric distribution

\end{document}
