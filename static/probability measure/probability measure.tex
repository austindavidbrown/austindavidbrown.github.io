\documentclass[12pt, reqno]{amsart}
\usepackage[utf8]{inputenc}
\usepackage{amsmath, amsfonts, amssymb, amsthm, setspace}
\numberwithin{equation}{section}
\parindent 0mm
\onehalfspacing
\addtolength{\textwidth}{2 truecm}
\setlength{\hoffset}{-1 truecm}

\theoremstyle{definition}
\newtheorem{theorem}{Theorem}
\newtheorem*{theorem*}{Theorem}
\newtheorem{proposition}{Proposition}
\newtheorem*{proposition*}{Proposition}
\newtheorem{lemma}{Lemma}
\newtheorem{corollary}{Corollary}
\newtheorem{claim}{Claim}

\theoremstyle{definition}
\newtheorem{definition}{Definition}
\newtheorem*{definition*}{Definition}
\newtheorem{remark}{Remark}
\newtheorem{example}{Example}
\newtheorem*{example*}{Example}
\newtheorem{problem}{Problem}

\newcommand{\e}{\epsilon}
\newcommand{\R}{\mathbb{R}}
\newcommand{\N}{\mathbb{N}}
\newcommand{\Z}{\mathbb{Z}}
\newcommand{\w}{\omega}
\let\oldemptyset\emptyset
\let\emptyset\varnothing

\newcommand{\intersect}{\bigcap}
\newcommand{\union}{\bigcup}
\newcommand{\disjointunion}{\coprod} 
\newcommand{\powerset}{\mathcal{P}}
\newcommand{\dist}{\sim}
\renewcommand{\P}{\text{P}}
\newcommand{\E}{\text{E}}

\begin{document}

\title{Measure and Probability Measure}
\author{Austin David Brown}
\maketitle

\begin{abstract}
Some personal notes on measure and probability measure. 
\end{abstract}

\section{Axioms of Measure}

Probability is a special type of volume ranging from $[0, 1]$. In order to handle volume in mathematics, there is Measure Theory. Probability is a special type of measure.

There is an issue with uncountable sets (Banach-Tarski Paradox), and hence we need to restrict the type of sets for which we can define volume for.

\begin{definition*}
A $\sigma$-algebra $\mathcal{M}$ is a collection of subsets of a space $\Omega$ satisfying the following axioms:
\begin{itemize}
\item (Nonempty) $\mathcal{M} \not= \emptyset$.
\item (Closure under compliments) If $A \in \mathcal{M}$ then $A^c \in \mathcal{M}$.
\item (Closure under countable unions) If $\{A_i\}_{i} \in \mathcal{M}$ then $\bigcup_{i} A_i \in \mathcal{M}$.
\end{itemize}
\end{definition*}

The sigma algebra is a formalization of the properties of measurable sets since
the compliment of a measurable set is measurable and the countable union of measurable sets is measurable.

\begin{definition*}
Let $X$ be a topological space. The $\sigma$-algebra $\mathcal{B}_X$ generated by the open sets is the Borel $\sigma$-algebra.
\end{definition*}

The idea is volume on a set should be non-negative, nothing should have volume $0$, and if we add more stuff to the set, the volume should grow additively. 

\begin{definition*}
A Measure is a map $\mu : \mathcal{M} \to [0, \infty]$ where $\mathcal{M}$ is a $\sigma$-algebra and satisfies the following axioms:
\begin{itemize}
\item $\mu(\emptyset) = 0$
\item Countable additivity: $\mu(\bigcup A_i) = \sum_{i} \mu(A_i)$
\end{itemize}
\end{definition*}

\begin{definition*}
A Probability Measure is a measure $P : \mathcal{M} \to [0, 1]$ where $\mathcal{M}$ is a $\sigma$-algebra, and $P(\mathcal{M}) = 1$.
\end{definition*}

\begin{example*}
Let $X$ be a nonempty set. By the axioms, the power set $\powerset(X)$ is a $\sigma$-algebra.
\end{example*}

\begin{example*}
Let $\Omega = \{ 0, 1 \}$. By the axioms, the power set $\powerset(X)$ of $\Omega$ is a $\sigma$-algebra and $p(\w) = 1/2$ for $\w \in \Omega$ is a probability measure on the power set. This example extends to the $n$ case.
\end{example*}

\begin{example*}
Use the example above to measure probability of the top side of dice rolls.
\end{example*}

\section{Properties of Measure}

Let $\mu$ be a measure on $(X, \mathcal{M})$ where $\mathcal{M}$ is a $\sigma$-algebra for the set $X$.

\begin{proposition*}
(Monotonicity)
If $A, B \in \mathcal{M}$ with $A \subset B$,
then $\mu(A) \le \mu(B).$
\end{proposition*}

\begin{proof}
$\mu(A) \le \mu(A) + \mu(B \setminus A) = \mu(A \coprod B \setminus A) = \mu(B).$
\end{proof}

\begin{proposition*}
(Subadditivity)
If $\{ E_k \}_k \subset \mathcal{M}$,
then $\mu(\union_{k \in \N} E_k) \le \sum_{k \in \N} \mu(E_k).$
\end{proposition*}

\begin{proof}
The idea is to put the problem in terms of a disjoint union and use the additive property of measure.
Indeed,
\[
\union_{k = 1}^\infty E_k = \coprod_{k = 1}^\infty \left( E_k \setminus \union_{j = 1}^{k - 1} E_j \right)
\]
and the right-hand side is a disjoint union.
By the additive property of measure,
\[
\mu \left( E_k \setminus \union_{j = 1}^{k - 1} E_j \right) = \mu(E_k) - \mu \left(\union_{j = 1}^{k - 1} E_j \right).
\]

Applying the additive property of measure again and using the above results, we have
\begin{align*}
\mu \left( \union_{k = 1}^\infty E_k \right)
&= \mu \left( \coprod_{k = 1}^\infty \left( E_k \setminus \union_{j = 1}^{k - 1} E_j \right) \right) \\
&= \sum_{k = 1}^\infty \mu \left( E_k \setminus \union_{j = 1}^{k - 1} E_j \right) \\
&= \sum_{k = 1}^\infty \mu(E_k ) - \mu \left( \union_{j = 1}^{k - 1} E_j \right) \\
&\le \sum_{k = 1}^\infty \mu(E_k ).
\end{align*}

\end{proof}

\begin{proposition*}
(Continuity from Below)
If $\{ E_k \}_k \subset \mathcal{M}$ with $E_1 \subset \cdots$,
then $\mu(\union_{k\in\N} E_k) = \lim_{n \to \infty} \mu(E_n)$.
\end{proposition*}

\begin{proof}
The idea is to construct a sequence in $\mathcal{M}$ whose union telescopes and is disjoint.
Define $(G_k)_k$  = $(E_k \setminus E_{k - 1})_{k \in \N, k \ge 2}$ and $G_1 = \emptyset$.
By construction, $\union_{k = 1}^n G_k = E_n \setminus E_0 = E_n$ since it telescopes, and is disjoint $\disjointunion_{k \in \N} G_k$.
By properties of measures,
$$
\mu(\lim_{n \to \infty} \union_{k = 1}^n G_k)
= \lim_{n \to \infty} \sum_{k = 1}^n \mu(G_k)
= \lim_{n \to \infty} \mu(\union_{k = 1}^n G_k)
= \lim_{n \to \infty} \mu(E_n).
$$
\end{proof}

\begin{proposition*}
(Continuity from Above) 
If $\{ E_k \}_k \subset \mathcal{M}$,
$\mu(E_k) < \infty$,
and $E_1 \supset \cdots$,
then $\mu(\intersect_{k\in\N} E_k) = \lim_{n \to \infty} \mu(E_n)$.
\end{proposition*}

\begin{proof}

The idea is to put this intersection in terms of a disjoint union and use additive property of measure.
Hence, by the additive property of measure,
\[
\mu(E_1)
= \mu \left(E_1 \setminus \intersect_{k \in \N} E_k \coprod \intersect_{k \in \N} E_k \right) \\
= \mu(E_1 \setminus \intersect_{k \in \N} E_k) + \mu(\intersect_{k \in \N} E_k).
\]
Now, $\{ E_1 \setminus E_k\}_k \subset \mathcal{M}$, and $E_1 \setminus E_2 \subset \cdots$, so we apply the continuity from above property, and the additivity property of measure. Hence, we have
\begin{align*}
\mu(E_1 \setminus \intersect_{k \in \N} E_k)
&= \mu(\union_{k \in \N} E_1 \setminus E_k) \\
&= \lim_{k \to \infty} \mu(E_1 \setminus E_k) \\
&= \lim_{k \to \infty} \mu(E_1) - \mu(E_k) \\
&= \mu(E_1) - \lim_{k \to \infty}\mu(E_k).
\end{align*}

Since $\mu(E_1) < \infty$ and combining the above results,
\[
\mu(\intersect_{k \in \N} E_k) = \lim_{k \to \infty}\mu(E_k).
\]
\end{proof}

\begin{proposition*}
If $A, B \in \mathcal{M}$,
then $\mu(A \union B) = \mu(A) + \mu(B) - \mu(A \intersect B).$
\end{proposition*}

\begin{proof}
The idea is to put this union in terms of a disjoint union and use additive property of measure.
Since $A \union B = A \union B \setminus A$, by properties of measure,
$$
\mu(A \union B)
= \mu(A \union B \setminus A)
= \mu(A) + \mu(B \setminus A).
$$
Since $B \setminus A = B\setminus A \intersect B$,
$$
\mu(B \setminus A)
= \mu(B \setminus A \intersect B) 
= \mu(B) - \mu(A \intersect B).
$$
Combine both results.

\end{proof}

\begin{thebibliography}{1}
\bibitem{folland}
G. Folland. \textit{Real Analysis}. Wiley, New York. 1999.

\bibitem{durett}
R. Durett. \textit{Probability: Theory and Examples}. Cambridge, New York. 2010.
\end{thebibliography}

\end{document}
